% Options for packages loaded elsewhere
\PassOptionsToPackage{unicode}{hyperref}
\PassOptionsToPackage{hyphens}{url}
\PassOptionsToPackage{dvipsnames,svgnames,x11names}{xcolor}
%
\documentclass[
]{report}

\usepackage{amsmath,amssymb}
\usepackage{iftex}
\ifPDFTeX
  \usepackage[T1]{fontenc}
  \usepackage[utf8]{inputenc}
  \usepackage{textcomp} % provide euro and other symbols
\else % if luatex or xetex
  \usepackage{unicode-math}
  \defaultfontfeatures{Scale=MatchLowercase}
  \defaultfontfeatures[\rmfamily]{Ligatures=TeX,Scale=1}
\fi
\usepackage{lmodern}
\ifPDFTeX\else  
    % xetex/luatex font selection
\fi
% Use upquote if available, for straight quotes in verbatim environments
\IfFileExists{upquote.sty}{\usepackage{upquote}}{}
\IfFileExists{microtype.sty}{% use microtype if available
  \usepackage[]{microtype}
  \UseMicrotypeSet[protrusion]{basicmath} % disable protrusion for tt fonts
}{}
\makeatletter
\@ifundefined{KOMAClassName}{% if non-KOMA class
  \IfFileExists{parskip.sty}{%
    \usepackage{parskip}
  }{% else
    \setlength{\parindent}{0pt}
    \setlength{\parskip}{6pt plus 2pt minus 1pt}}
}{% if KOMA class
  \KOMAoptions{parskip=half}}
\makeatother
\usepackage{xcolor}
\setlength{\emergencystretch}{3em} % prevent overfull lines
\setcounter{secnumdepth}{-\maxdimen} % remove section numbering
% Make \paragraph and \subparagraph free-standing
\ifx\paragraph\undefined\else
  \let\oldparagraph\paragraph
  \renewcommand{\paragraph}[1]{\oldparagraph{#1}\mbox{}}
\fi
\ifx\subparagraph\undefined\else
  \let\oldsubparagraph\subparagraph
  \renewcommand{\subparagraph}[1]{\oldsubparagraph{#1}\mbox{}}
\fi


\providecommand{\tightlist}{%
  \setlength{\itemsep}{0pt}\setlength{\parskip}{0pt}}\usepackage{longtable,booktabs,array}
\usepackage{calc} % for calculating minipage widths
% Correct order of tables after \paragraph or \subparagraph
\usepackage{etoolbox}
\makeatletter
\patchcmd\longtable{\par}{\if@noskipsec\mbox{}\fi\par}{}{}
\makeatother
% Allow footnotes in longtable head/foot
\IfFileExists{footnotehyper.sty}{\usepackage{footnotehyper}}{\usepackage{footnote}}
\makesavenoteenv{longtable}
\usepackage{graphicx}
\makeatletter
\def\maxwidth{\ifdim\Gin@nat@width>\linewidth\linewidth\else\Gin@nat@width\fi}
\def\maxheight{\ifdim\Gin@nat@height>\textheight\textheight\else\Gin@nat@height\fi}
\makeatother
% Scale images if necessary, so that they will not overflow the page
% margins by default, and it is still possible to overwrite the defaults
% using explicit options in \includegraphics[width, height, ...]{}
\setkeys{Gin}{width=\maxwidth,height=\maxheight,keepaspectratio}
% Set default figure placement to htbp
\makeatletter
\def\fps@figure{htbp}
\makeatother

\makeatletter
\makeatother
\makeatletter
\makeatother
\makeatletter
\@ifpackageloaded{caption}{}{\usepackage{caption}}
\AtBeginDocument{%
\ifdefined\contentsname
  \renewcommand*\contentsname{Table of contents}
\else
  \newcommand\contentsname{Table of contents}
\fi
\ifdefined\listfigurename
  \renewcommand*\listfigurename{List of Figures}
\else
  \newcommand\listfigurename{List of Figures}
\fi
\ifdefined\listtablename
  \renewcommand*\listtablename{List of Tables}
\else
  \newcommand\listtablename{List of Tables}
\fi
\ifdefined\figurename
  \renewcommand*\figurename{Figure}
\else
  \newcommand\figurename{Figure}
\fi
\ifdefined\tablename
  \renewcommand*\tablename{Table}
\else
  \newcommand\tablename{Table}
\fi
}
\@ifpackageloaded{float}{}{\usepackage{float}}
\floatstyle{ruled}
\@ifundefined{c@chapter}{\newfloat{codelisting}{h}{lop}}{\newfloat{codelisting}{h}{lop}[chapter]}
\floatname{codelisting}{Listing}
\newcommand*\listoflistings{\listof{codelisting}{List of Listings}}
\makeatother
\makeatletter
\@ifpackageloaded{caption}{}{\usepackage{caption}}
\@ifpackageloaded{subcaption}{}{\usepackage{subcaption}}
\makeatother
\makeatletter
\@ifpackageloaded{tcolorbox}{}{\usepackage[skins,breakable]{tcolorbox}}
\makeatother
\makeatletter
\@ifundefined{shadecolor}{\definecolor{shadecolor}{rgb}{.97, .97, .97}}
\makeatother
\makeatletter
\makeatother
\makeatletter
\makeatother
\ifLuaTeX
  \usepackage{selnolig}  % disable illegal ligatures
\fi
\IfFileExists{bookmark.sty}{\usepackage{bookmark}}{\usepackage{hyperref}}
\IfFileExists{xurl.sty}{\usepackage{xurl}}{} % add URL line breaks if available
\urlstyle{same} % disable monospaced font for URLs
\hypersetup{
  pdftitle={Master SOP},
  colorlinks=true,
  linkcolor={blue},
  filecolor={Maroon},
  citecolor={Blue},
  urlcolor={Blue},
  pdfcreator={LaTeX via pandoc}}

\title{Master SOP}
\author{}
\date{}

\begin{document}
\maketitle
\ifdefined\Shaded\renewenvironment{Shaded}{\begin{tcolorbox}[enhanced, sharp corners, borderline west={3pt}{0pt}{shadecolor}, interior hidden, breakable, boxrule=0pt, frame hidden]}{\end{tcolorbox}}\fi

\hypertarget{description}{%
\section{Description}\label{description}}

This document contains the standard operating procedures for the
Northfield Family Area YMCA.

\hypertarget{birthday-parties}{%
\section{Birthday Parties}\label{birthday-parties}}

\hypertarget{member}{%
\subsubsection{Member}\label{member}}

\begin{enumerate}
\def\labelenumi{\arabic{enumi}.}
\tightlist
\item
  YMCA Member or Community Member registers online for a Birthday Party
\end{enumerate}

\hypertarget{system}{%
\subsubsection{System}\label{system}}

\begin{enumerate}
\def\labelenumi{\arabic{enumi}.}
\setcounter{enumi}{1}
\tightlist
\item
  An email is automatically sent to Heidi alerting her that someone has
  registered for a party.
\end{enumerate}

\hypertarget{heidi}{%
\subsubsection{Heidi}\label{heidi}}

\begin{enumerate}
\def\labelenumi{\arabic{enumi}.}
\setcounter{enumi}{2}
\item
  Log into Daxko. In the Child Care tab, click on manage programs. Then,
  click on Birthday Party program- Q\&A for all, expand view, and on the
  answers for the registrant.
\item
  The type of party, time, and date are now visible. This information
  should be added to calendars.
\item
  When adding to Community Room Calendar write: Reserved for a Birthday
  Party
\item
  When adding to the Gym Calendar write: 1/2 Gym Reserved for a Birthday
  Party
\end{enumerate}

\hypertarget{anne}{%
\subsubsection{Anne}\label{anne}}

\begin{enumerate}
\def\labelenumi{\arabic{enumi}.}
\setcounter{enumi}{6}
\tightlist
\item
  When adding to the Pool Calendar write: Birthday Party Scheduled- Open
  Swim Still Available. (Please note that a birthday party will only be
  added to the pool calendar when there will be more than 20 swimmers
  attending)
\end{enumerate}

\hypertarget{heidi-1}{%
\subsubsection{Heidi}\label{heidi-1}}

\begin{enumerate}
\def\labelenumi{\arabic{enumi}.}
\setcounter{enumi}{7}
\item
  Add the fill in information in the Gym or Pool Party info email and
  email to party parent/ guardian.
\item
  Request the t-shirt size and place it on top of the birthday party bin
  with a note that includes: Birthday child's name \& party date.
\item
  If a gym party was chosen, gather requested equipment into a rolling
  blue garbage bin. Keep this bin in the cage prior to the party with a
  note that includes: Birthday child's name \& party date.
\item
  Send out an email to lead staff team with the following information:
  subject: Birthday Party Scheduled. Body text includes: Date, time,
  chosen activity, approx. number of children and adults, and birthday
  child's name.
\end{enumerate}

\hypertarget{anne-1}{%
\subsubsection{Anne}\label{anne-1}}

\begin{enumerate}
\def\labelenumi{\arabic{enumi}.}
\setcounter{enumi}{11}
\item
  If there is a gym party, communicate with the Wellness team that the
  divider must be put down on birthday party date. As well, the Wellness
  team must get the blue garbage bin with equipment out of the cage and
  roll it to the back half of the gym before the party begins.
\item
  If there is a pool party, communicate with the Lifeguard staff, and
  schedule extra lifeguards for that day if needed.
\end{enumerate}

\hypertarget{jesse}{%
\subsubsection{Jesse}\label{jesse}}

\begin{enumerate}
\def\labelenumi{\arabic{enumi}.}
\setcounter{enumi}{13}
\tightlist
\item
  Communicate with Janitorial staff that after the birthday party they
  will need to check the community room, and other spaces used during
  the party. If there are damages sustained from the party, or extra
  mess that is not cleaned, janitorial staff will take photos and let
  Jesse know. Communicate with Heidi so she can charge additional fees
  when needed.
\end{enumerate}

\hypertarget{suzanne}{%
\subsubsection{Suzanne}\label{suzanne}}

\begin{enumerate}
\def\labelenumi{\arabic{enumi}.}
\setcounter{enumi}{14}
\item
  Communicate with member services team that when the party comes in: a
  vacuum, party t-shirt for the birthday child, and the birthday party
  bin should be provided to the registrants.
\item
  Communicate with member services team that everyone that comes to the
  party must be checked into the facility. Party attendees will have
  multiple options of how they can enter their information.
\end{enumerate}

\begin{itemize}
\tightlist
\item
  If they are members, they can just stop at the front desk and check
  in.
\item
  If they are not members, they can go online on their own and set up a
  community member account.
\item
  Non-members can fill out the birthday party form prior to the party
  and bring it to the MS desk on entry
\item
  Non-members can request a birthday party form and fill it out upon
  arrival.
\end{itemize}

\begin{enumerate}
\def\labelenumi{\arabic{enumi}.}
\setcounter{enumi}{16}
\tightlist
\item
  Communicate with member service team that upon arrival, registrants
  will need to make sure all party attendants are checked into Daxko.
  Party attendants will be given instructions prior to the party on how
  to enter/update their information in Daxko. Ms will also need to take
  photos of all party attendees that do not already have one in the
  system.
\end{enumerate}

\hypertarget{heidi-2}{%
\subsubsection{Heidi}\label{heidi-2}}

\begin{enumerate}
\def\labelenumi{\arabic{enumi}.}
\setcounter{enumi}{17}
\tightlist
\item
  On Monday or Tuesday of the following week, email out a Thank You to
  the party registrant for celebrating at the YMCA.
\end{enumerate}



\end{document}
