% Options for packages loaded elsewhere
\PassOptionsToPackage{unicode}{hyperref}
\PassOptionsToPackage{hyphens}{url}
\PassOptionsToPackage{dvipsnames,svgnames,x11names}{xcolor}
%
\documentclass[
]{report}

\usepackage{amsmath,amssymb}
\usepackage{iftex}
\ifPDFTeX
  \usepackage[T1]{fontenc}
  \usepackage[utf8]{inputenc}
  \usepackage{textcomp} % provide euro and other symbols
\else % if luatex or xetex
  \usepackage{unicode-math}
  \defaultfontfeatures{Scale=MatchLowercase}
  \defaultfontfeatures[\rmfamily]{Ligatures=TeX,Scale=1}
\fi
\usepackage{lmodern}
\ifPDFTeX\else  
    % xetex/luatex font selection
\fi
% Use upquote if available, for straight quotes in verbatim environments
\IfFileExists{upquote.sty}{\usepackage{upquote}}{}
\IfFileExists{microtype.sty}{% use microtype if available
  \usepackage[]{microtype}
  \UseMicrotypeSet[protrusion]{basicmath} % disable protrusion for tt fonts
}{}
\makeatletter
\@ifundefined{KOMAClassName}{% if non-KOMA class
  \IfFileExists{parskip.sty}{%
    \usepackage{parskip}
  }{% else
    \setlength{\parindent}{0pt}
    \setlength{\parskip}{6pt plus 2pt minus 1pt}}
}{% if KOMA class
  \KOMAoptions{parskip=half}}
\makeatother
\usepackage{xcolor}
\setlength{\emergencystretch}{3em} % prevent overfull lines
\setcounter{secnumdepth}{-\maxdimen} % remove section numbering
% Make \paragraph and \subparagraph free-standing
\ifx\paragraph\undefined\else
  \let\oldparagraph\paragraph
  \renewcommand{\paragraph}[1]{\oldparagraph{#1}\mbox{}}
\fi
\ifx\subparagraph\undefined\else
  \let\oldsubparagraph\subparagraph
  \renewcommand{\subparagraph}[1]{\oldsubparagraph{#1}\mbox{}}
\fi


\providecommand{\tightlist}{%
  \setlength{\itemsep}{0pt}\setlength{\parskip}{0pt}}\usepackage{longtable,booktabs,array}
\usepackage{calc} % for calculating minipage widths
% Correct order of tables after \paragraph or \subparagraph
\usepackage{etoolbox}
\makeatletter
\patchcmd\longtable{\par}{\if@noskipsec\mbox{}\fi\par}{}{}
\makeatother
% Allow footnotes in longtable head/foot
\IfFileExists{footnotehyper.sty}{\usepackage{footnotehyper}}{\usepackage{footnote}}
\makesavenoteenv{longtable}
\usepackage{graphicx}
\makeatletter
\def\maxwidth{\ifdim\Gin@nat@width>\linewidth\linewidth\else\Gin@nat@width\fi}
\def\maxheight{\ifdim\Gin@nat@height>\textheight\textheight\else\Gin@nat@height\fi}
\makeatother
% Scale images if necessary, so that they will not overflow the page
% margins by default, and it is still possible to overwrite the defaults
% using explicit options in \includegraphics[width, height, ...]{}
\setkeys{Gin}{width=\maxwidth,height=\maxheight,keepaspectratio}
% Set default figure placement to htbp
\makeatletter
\def\fps@figure{htbp}
\makeatother

\makeatletter
\makeatother
\makeatletter
\makeatother
\makeatletter
\@ifpackageloaded{caption}{}{\usepackage{caption}}
\AtBeginDocument{%
\ifdefined\contentsname
  \renewcommand*\contentsname{Table of contents}
\else
  \newcommand\contentsname{Table of contents}
\fi
\ifdefined\listfigurename
  \renewcommand*\listfigurename{List of Figures}
\else
  \newcommand\listfigurename{List of Figures}
\fi
\ifdefined\listtablename
  \renewcommand*\listtablename{List of Tables}
\else
  \newcommand\listtablename{List of Tables}
\fi
\ifdefined\figurename
  \renewcommand*\figurename{Figure}
\else
  \newcommand\figurename{Figure}
\fi
\ifdefined\tablename
  \renewcommand*\tablename{Table}
\else
  \newcommand\tablename{Table}
\fi
}
\@ifpackageloaded{float}{}{\usepackage{float}}
\floatstyle{ruled}
\@ifundefined{c@chapter}{\newfloat{codelisting}{h}{lop}}{\newfloat{codelisting}{h}{lop}[chapter]}
\floatname{codelisting}{Listing}
\newcommand*\listoflistings{\listof{codelisting}{List of Listings}}
\makeatother
\makeatletter
\@ifpackageloaded{caption}{}{\usepackage{caption}}
\@ifpackageloaded{subcaption}{}{\usepackage{subcaption}}
\makeatother
\makeatletter
\@ifpackageloaded{tcolorbox}{}{\usepackage[skins,breakable]{tcolorbox}}
\makeatother
\makeatletter
\@ifundefined{shadecolor}{\definecolor{shadecolor}{rgb}{.97, .97, .97}}
\makeatother
\makeatletter
\makeatother
\makeatletter
\makeatother
\ifLuaTeX
  \usepackage{selnolig}  % disable illegal ligatures
\fi
\IfFileExists{bookmark.sty}{\usepackage{bookmark}}{\usepackage{hyperref}}
\IfFileExists{xurl.sty}{\usepackage{xurl}}{} % add URL line breaks if available
\urlstyle{same} % disable monospaced font for URLs
\hypersetup{
  colorlinks=true,
  linkcolor={blue},
  filecolor={Maroon},
  citecolor={Blue},
  urlcolor={Blue},
  pdfcreator={LaTeX via pandoc}}

\author{}
\date{}

\begin{document}
\ifdefined\Shaded\renewenvironment{Shaded}{\begin{tcolorbox}[breakable, interior hidden, borderline west={3pt}{0pt}{shadecolor}, boxrule=0pt, sharp corners, enhanced, frame hidden]}{\end{tcolorbox}}\fi

\hypertarget{basic-information-overview}{%
\chapter{Basic Information \&
Overview}\label{basic-information-overview}}

\begin{itemize}
\item
  You are the first and last experience for a member; make sure to greet
  them as they come in and leave. Friendliness makes a difference.
\item
  Members in the building come first. Voicemails can be checked and we
  can get back to phone calls.
\item
  Personal cell phone use is restricted when at the front desk. If you
  need to make a call, walkie the MOD to cover while you step in the
  back.
\item
  Do not leave the desk unattended. If you need to use the restroom, get
  a drink, or grab food, walkie the MOD up to cover while you step away.
\item
  Supervisors determine when breaks should be taken in accordance with
  staffing and program requirements. Breaks cannot be added to extend a
  meal period or used to end work early.
\item
  A short duration rest break is available within each period of four
  consecutive work hours. This includes bathroom breaks, phone calls and
  any other times you step away from the front desk. Food can be eaten
  up at the front desk if needed, but make sure this is done at slower
  times during your shift.
\item
  If scheduled to work eight or more consecutive hours, a 30-minute
  unpaid meal break is required. MOD is responsible for coverage of
  these breaks.
\item
  There are mailboxes for different staff members in the back behind
  Member Services in case you need to give forms or other info to
  someone. The mailbox for each individual is the box to the right of
  their name.
\item
  New information and updates are put in the MS information Binder. You
  should also get emails, so please make sure to check your email, and
  the binger, during your shift.
\end{itemize}

\hypertarget{daxko-training-site-information}{%
\subsubsection{Daxko Training Site
Information}\label{daxko-training-site-information}}

Each staff member has their own log in to the Daxko Training site.

The training site can be found if you click Daxko Operations (Training).
Keep this page separate from the live version of Daxko. All edits made
on the training site do not actually apply and are purely for practice.

\hypertarget{member-service-staff-email}{%
\subsubsection{Member Service Staff
Email}\label{member-service-staff-email}}

All MS Staff have their own email and should check it at the beginning
and throughout their shift (at least hourly). This can be accessed
through the Outlook application.

This email should only be used for internal purposes, which include:

\begin{itemize}
\tightlist
\item
  Updates for Member Services Staff
\item
  To communicate with Lead Team Staff who are not in the building
\item
  Internal YMCA business
\end{itemize}

Do NOT use this email to communicate directly with members.

Please use email to communicate with other internal staff members
instead of leaving them post-it notes, as post-it's can get lost.

Make sure to include member's name, contact information and details
regarding reason for email when sending a message to a staff member.

Please make sure to send any Member Services/Membership questions or
issue to both Suzanne \& Megan, so that one of us can answer.

\hypertarget{schedule-time-off-requests}{%
\subsubsection{Schedule \& Time Off
Requests}\label{schedule-time-off-requests}}

All communications regarding schedules will be sent to staff's personal
email. Please make sure you can receive emails from
megan@northfieldymca.org.

Schedules are created at least one month at a time and should be sent
out mid-way through the previous month.

Time off requests should be submitted at least 2 months in advance to
help with the scheduling process to make sure shifts are covered. If
anything comes up unexpectedly once the schedule is complete, then you
are responsible for finding coverage, by either switching with someone
or finding someone to cover for you. Contact information for Member
Services staff is in the Information Binder.

Timekeeping \& Exception Forms

You should use the ADP app to clock in and out. This is to be done in
the laundry room when you pick up/drop off your walkie. There is a Quick
Clock on the app that can be used without having to sign in, otherwise,
once you sign in you will click on either Start Work, or End Work. If
you miss punching in or out, there are exception forms that can be
filled out with the missed time. These are in the Laundry Room on the
whiteboard. Fill out the form and put it in Suzanne's mailbox before the
end of the pay period. Pay periods are the 1st-15th, and the 15th-Last
day. You will get paid on the 15th and the Last day of the month.

\hypertarget{confidential-information}{%
\subsubsection{Confidential
Information}\label{confidential-information}}

All information in Daxko is always confidential.

Daxko and membership information should only be given to adults (18+)
that are listed on the unit. * There are some situations where it might
seem ok to give out personal information to someone, but please remember
that giving personal info to anyone that is not listed on the unit, and
over 18, is not approved.

Daxko information in a membership, especially billing and scholarship
information, should never be given to minors. Occasionally a minor will
come in to use the Y, and the membership in not currently active. In
this case, do no give out billing info or scholarship info to the minor.
Make sure that you have the parent/guardian's correct contact
information and email the admin team so they can follow up with the
parent/guardian.

You should never give out any youth program participant information to
any adult that is not listed on the child's unit.

All youth program participant registration info is confidential, and
only to be given out to the parent/guardian that is listed on the unit.

\begin{itemize}
\item
  There are instances in youth programming where a parent/guardian may
  call asking about a child's participation in a program. If that
  parent/guardian is not listed on the unit, they may not be given that
  information. This may be a situation where he/she is not allowed to
  see the child. In this case, always remember to check notes on the
  unit.
\item
  When a parent calls to ask for information about a child's
  registration, please make sure to verify the identity of the adult
  first.
\end{itemize}

If you are unsure of someone's identity, it is OK to ask for their date
of birth, address, email or phone number to verify.

\hypertarget{insurance-information}{%
\subsubsection{Insurance Information}\label{insurance-information}}

Members can sign up to get reimbursed from their insurance company if
they come a certain number of times a month, depending on the company.

Insurance forms are in the drawer under the front desk. There are
separate folders for each of the insurance companies.

Insurance forms should be filled out, initialed and signed. Make a copy
of the front of their insurance card and they need to attach a voided
check.

Completed forms, which include the Enrollment form, voided check and
copy of insurance card, can be put in Megan's mailbox.

We do not take Blue Cross Blue Shield.

We do not participate in SilverSneakers, Renew Active, One Pass, and
Active \& Fit.

We do have a Silver \& Fit option, applications are in the drawer.

\hypertarget{safety-protocols}{%
\subsubsection{Safety Protocols}\label{safety-protocols}}

There is an Emergency Action Procedures binder in the front desk drawer.
This has all procedures that should be followed in different types of
emergencies.

The Safety committee will be performing drills throughout the year to
ensure that all members of staff know the emergency procedures.

Incident Reporting -- in the back of the EAP binder there is a sheet
that outlines what types of incidents need to be reported and how to do
that. There are blank Incident Reports in the back pocket of the EAP
binder.

There is a sheet in the back of the EAP binder that provides direction
on what to do when you need to clean up blood or other bodily fluids on
non-carpeted surfaces.

There is an emergency intercom for the pool deck on the wall behind the
front desk. This is so if there is an emergency in the pool the
lifeguard can contact the front quickly. If this goes off, follow the
intercom protocols that are above the intercom.

\hypertarget{self-check-in-station}{%
\subsubsection{Self-Check In Station}\label{self-check-in-station}}

There is a Self-Check In station set up at the 2nd front desk computer,
this is to be used for active members only, anyone else coming in for
any other reason should check in with Member Services staff. There are
signs up front to help direct members.

Member Services staff should still pay attention to everyone coming in
the building, and should be where most people are checking in. The
self-check in station is a tool to help with the busy times when there
is not double coverage up at the front. This is not an excuse to not pay
attention, you should still be greeting everyone when they come in.

Member Services should monitor the self-check in station to make sure
those checking in are active.

Regarding rosters: You can use the ``Who is Here?'' list to see who is
in the building to compare to the roster. You can call a MOD up to help
monitor when you know a class is starting if that is helpful.

\hypertarget{tour-information}{%
\subsubsection{Tour Information}\label{tour-information}}

If someone calls and requests a tour, let them know they can come in
anytime for a tour, or if they want to set up a time they can do that as
well. If they set up a specific day/time, email the Member Services
staff member who will be on duty at that time and let them know the
day/time that the visitor will be coming in, along with their name.

If someone just walks in and requests a tour, let them know you will get
someone to show them around. Have the visitor fill out the Membership
Inquiry form, which is on the small clipboards at the front of the
drawer.

If there is only 1 member services staff member, then walkie Wellness to
see if they can come to the front and give a tour.

If there are 2 member services staff members, one of you can give the
tour.

\hypertarget{department-contact-list}{%
\section{Department Contact List}\label{department-contact-list}}

Member Services: Suzanne

\begin{itemize}
\tightlist
\item
  Membership/ Scholarship questions
\item
  Daxko or Administrative concerns
\end{itemize}

Health \& Wellness and Aquatics: Erik \& Anne

\begin{itemize}
\tightlist
\item
  Wellness Floor: Erik
\item
  Group Exercise: Erik
\item
  Swimming Lessons: Anne
\item
  Open Swim/ Lap Swim: Anne
\item
  Personal Training: Anne
\end{itemize}

Youth Development: Heidi \& Izzy * Child Watch * Birthday Parties * Camp
* Youth in Government

Building \& Grounds: Jesse

\begin{itemize}
\tightlist
\item
  Maintenance
\item
  Deliveries
\item
  Building projects
\end{itemize}

\hypertarget{accepting-ordered-items-and-mail-from-vendor}{%
\chapter{Accepting Ordered Items and Mail from
Vendor}\label{accepting-ordered-items-and-mail-from-vendor}}

\hypertarget{vendor}{%
\subsubsection{Vendor}\label{vendor}}

\begin{enumerate}
\def\labelenumi{\arabic{enumi}.}
\item
  A vendor/UPS/Mail worker will enter the building with items to be
  dropped off.
\item
  Vendors will give item/s to Member Services team members.
\end{enumerate}

\hypertarget{ms-team-member}{%
\subsubsection{MS Team Member}\label{ms-team-member}}

\begin{enumerate}
\def\labelenumi{\arabic{enumi}.}
\setcounter{enumi}{2}
\item
  Member Services members will put all mail into Rachel's mailbox and
  all large items on the floor in front of the file cabinets below the
  mailboxes.
\item
  Please make sure that all items are on the floor in this area so they
  are not in the sight line of entering members.
\item
  If there are any large items, Member Services team members will email
  program directors to alert them of the delivery.
\end{enumerate}

\hypertarget{admin}{%
\subsubsection{Admin}\label{admin}}

\begin{enumerate}
\def\labelenumi{\arabic{enumi}.}
\setcounter{enumi}{5}
\tightlist
\item
  Administrative assistant will open all packages during shift and alert
  necessary staff of delivery
\end{enumerate}

\hypertarget{accessing-member-services-calendar}{%
\chapter{Accessing Member Services
Calendar}\label{accessing-member-services-calendar}}

\hypertarget{ms-team-member-1}{%
\subsubsection{MS Team Member}\label{ms-team-member-1}}

\begin{enumerate}
\def\labelenumi{\arabic{enumi}.}
\tightlist
\item
  Log into the front desk computer
\item
  Open outlook
\item
  Click on the calendar icon on the left-hand side.
\item
  Scroll down to All Group calendars
\item
  Check the box next to member services
\end{enumerate}

\hypertarget{adding-a-new-credit-card-or-bank-account}{%
\chapter{Adding a New Credit Card or Bank
Account}\label{adding-a-new-credit-card-or-bank-account}}

\hypertarget{ms-team-member-2}{%
\subsubsection{MS Team Member}\label{ms-team-member-2}}

\begin{enumerate}
\def\labelenumi{\arabic{enumi}.}
\tightlist
\item
  Open Daxko.
\item
  Click on the Membership tap at the top of the screen
\item
  Click search.
\item
  Type in ``last name, first name''
\item
  Click on the member's name
\item
  To edit, click on the pencil in the upper left-hand corner of the page
  where it says invoice.
\item
  Click on the ``add credit card button'', or use drop down to choose
  EFT.
\item
  Enter information requested. Always check to make sure it is spelled
  correctly, first and last names are capitalized, and the number and
  expiration date are correct.
\item
  Click Add Billing Method
\end{enumerate}

\hypertarget{monthly-dues-with-new-billing-method}{%
\subsubsection{Monthly dues with new billing
method}\label{monthly-dues-with-new-billing-method}}

\begin{enumerate}
\def\labelenumi{\arabic{enumi}.}
\setcounter{enumi}{9}
\tightlist
\item
  Click on the edit pencil in the upper left-hand corner of the page,
  where it says Bank Account or Credit Card.
\item
  Click Membership Dues.
\item
  Under Billing Method, click the arrow to choose the correct one.
\item
  Click on Membership Information to get back to the member page. It
  should now show the new payment option.
\end{enumerate}

\hypertarget{delete-the-old-billing-method}{%
\subsubsection{Delete the old billing
method}\label{delete-the-old-billing-method}}

\begin{enumerate}
\def\labelenumi{\arabic{enumi}.}
\setcounter{enumi}{13}
\tightlist
\item
  Click on the More button at the top right-hand of the screen.
\item
  Click on Update Billing Methods.
\item
  Click on the Trash Can next to the old form of payment.
\end{enumerate}

\hypertarget{adding-a-new-community-member}{%
\chapter{Adding a New Community
Member}\label{adding-a-new-community-member}}

\hypertarget{ms-team-member-3}{%
\subsubsection{MS Team Member}\label{ms-team-member-3}}

\begin{enumerate}
\def\labelenumi{\arabic{enumi}.}
\tightlist
\item
  Open Daxko
\item
  Click on Membership tab at the top of the screen
\item
  Click ``Search''
\item
  Type in last name, first name
\item
  If the member is already in Daxko click their name. If the person is
  not in Daxko skip to step 21
\end{enumerate}

\hypertarget{if-a-community-member}{%
\subsubsection{If a Community Member}\label{if-a-community-member}}

\begin{enumerate}
\def\labelenumi{\arabic{enumi}.}
\setcounter{enumi}{5}
\tightlist
\item
  Click on the pencil next to the word inactive to edit
\item
  Choose Community Member
\item
  Enter additional information if requested by Daxko, especially
  Birthdays
\item
  Click next
\item
  Choose which members on the account should be active. If they are
  purchasing an individual day pass, just 6. activate the person on the
  pass. If it is a family pass then all members can be activated.
\item
  Click next
\item
  A community Member has already signed a waiver if they have registered
  for a program in our system. You can type in the adult member's name
  and electronically sign the waiver for them at this step.
\item
  Click I agree
\item
  The system will set up this membership to automatically terminate in 1
  year, there will also be no recurring charges or joiners fee
\item
  Click next
\item
  Click on Membership Information tab at the top of the page
\item
  Check them in, if applicable.
\end{enumerate}

\hypertarget{if-inactive-member-on-active-unit}{%
\subsubsection{If Inactive Member On Active
Unit}\label{if-inactive-member-on-active-unit}}

\begin{enumerate}
\def\labelenumi{\arabic{enumi}.}
\setcounter{enumi}{17}
\tightlist
\item
  Keep the community member as inactive
\item
  Add a Yellow Alert above their name that says: Member ``First Name''
  is a Community member, please keep inactive due to active member in
  this unit
\item
  Check them in, if applicable
\end{enumerate}

\hypertarget{if-the-person-is-not-in-daxko}{%
\subsubsection{If The Person Is Not In
Daxko}\label{if-the-person-is-not-in-daxko}}

\begin{enumerate}
\def\labelenumi{\arabic{enumi}.}
\setcounter{enumi}{20}
\tightlist
\item
  Have the person fill out the Community Member/Program Participant
  form.\hspace{0pt}\hspace{0pt} We need to get their signature
\item
  Activate the person as a new member. See how to Activate a New Member
  SOP, choose ``community member'' as membership type
\item
  Check them in, if applicable.
\item
  Sign your initials in the top right corner of the completed form to
  note that it has been entered into Daxko. 25. Include the date.
\item
  File the form in the appropriate folder in the top drawer of the
  Membership File Cabinet.
\end{enumerate}

\hypertarget{be-our-guest-passes}{%
\chapter{Be Our Guest Passes}\label{be-our-guest-passes}}

\hypertarget{ms-team-member-4}{%
\subsubsection{MS Team Member}\label{ms-team-member-4}}

\begin{enumerate}
\def\labelenumi{\arabic{enumi}.}
\tightlist
\item
  If a guest comes in with a Be Our Guest Pass, they are able to use the
  facility
\item
  Have them fill out a Program Participant/ Community Member form. These
  are found in the appropriate folder in the drawer beneath the front
  desk
\item
  Add them as a Community Member if they are not in Daxko, or activate
  them as Community Member if they are
\item
  Click on Balance Due in the upper right-hand corner
\item
  Click on Add One-Time Fees button
\item
  Check the box next to Be Our Guest Pass. The amount will be zero
  dollars, but this is for tracking purposes
\item
  Click the Add Fees button at the bottom of the page. It will take you
  back to the Balance Due page. No actual payment needs to be taken as
  this is free
\item
  Go back to the Membership Information page
\item
  Check them in, no key fob is required
\item
  Add a note to their account that they came in on a Be Our Guest Pass
\item
  Sign initials and date the Community Member/ Program Participant form
\item
  Staple the Be Our Guest Pass to the form
\item
  Put form in the Be Our Guest folder in the top of the Membership File
  Cabinet
\end{enumerate}

\hypertarget{caretaker-check-in}{%
\chapter{Caretaker Check in}\label{caretaker-check-in}}

\hypertarget{ms-team-member-5}{%
\subsubsection{MS Team member}\label{ms-team-member-5}}

\begin{enumerate}
\def\labelenumi{\arabic{enumi}.}
\tightlist
\item
  A caretaker will come to the front desk to check in with their client
\end{enumerate}

\hypertarget{if-this-is-their-first-time-in}{%
\subsubsection{If This is Their First Time
in}\label{if-this-is-their-first-time-in}}

\begin{enumerate}
\def\labelenumi{\arabic{enumi}.}
\setcounter{enumi}{1}
\tightlist
\item
  Have the person fill out the Caretaker Form. Look them up in Daxko,
  follow the corresponding steps below
\item
  Note in the top right corner of the Caretaker Form that is has been
  entered, include your initials and the date
\item
  File the Caretaker Form in the appropriate folder in the top drawer of
  the Membership File Cabinet
\end{enumerate}

\hypertarget{if-person-is-not-in-daxko}{%
\subsubsection{If person is not in
Daxko}\label{if-person-is-not-in-daxko}}

\begin{enumerate}
\def\labelenumi{\arabic{enumi}.}
\setcounter{enumi}{4}
\tightlist
\item
  Activate the person as a new member. See how to activate a New Member
  SOP, choose community member as the membership type
\item
  Check them in
\item
  Add a yellow alert to their account saying: Caretaker for ``Client
  Name''
\end{enumerate}

\hypertarget{if-a-community-member-1}{%
\subsubsection{If a Community Member}\label{if-a-community-member-1}}

\begin{enumerate}
\def\labelenumi{\arabic{enumi}.}
\setcounter{enumi}{7}
\tightlist
\item
  Click on the pencil next to the word inactive to edit
\item
  Choose Community Member
\item
  Enter additional information if requested by Daxko, especially
  Brithdays
\item
  Click next
\item
  Choose which members should be active
\item
  Click next
\item
  You can type in the adult member's name and sign the waiver for them
  at this step
\item
  Click I agree
\item
  The system will set up this membership to automatically terminate in 1
  year, it will also say there is no charge
\item
  Click next
\item
  Click on Membership Information
\item
  Check them in
\item
  Add a yellow alert to their account saying: Caretaker for ``Client
  Name''
\end{enumerate}

\hypertarget{if-inactive-member-on-active-unit-1}{%
\subsubsection{If Inactive Member on Active
unit}\label{if-inactive-member-on-active-unit-1}}

\begin{enumerate}
\def\labelenumi{\arabic{enumi}.}
\setcounter{enumi}{20}
\tightlist
\item
  Keep the community member as inactive
\item
  Add a yellow alert above their name that says: Member ``First Name''
  is a caretaker for ``Client Name'', please keep inactive due to active
  member in this unit
\end{enumerate}

\hypertarget{if-active-community-member}{%
\subsubsection{If active community
member}\label{if-active-community-member}}

\begin{enumerate}
\def\labelenumi{\arabic{enumi}.}
\setcounter{enumi}{22}
\tightlist
\item
  Check them in
\item
  Add a yellow alert to their account saying: Caretaker for ``Client
  name''
\end{enumerate}



\end{document}
